This paper modeled the housing market by splitting it into three subcategories. Then on this model model compute simulations of it and lastly discus the results. The market will be divided from the supply side as follows: category one for subsidised public allocation (that will match agents with a FCFS queue), social advertised that includes all allocation that can be found publicly (as in web sites or social media), agents will have a probability to be matched to an apartment here and lately decide to take it or not and lastly accommodation found trough the help of landlord or real estate agency, agents will choose this category at last. Moreover the agents will have a maximum waiting time to find accommodation and after that they will perish and leave the market. Lastly the arrivals of houses and agents will be modeled with the help of Poisson processes. After the simulations were done we were able to observe that agents highly favorites the public subsidised sector. Another import point to remark is that the model showed that the system can reach an equilibrium where all agents have an apartment matched to them. However for that there should be more apartments available in the market than demand, otherwise agents will leave the market. Agents will leave the market even with a balanced market where there is exactly one apartment for each agent and exactly one agent for each apartment. Moreover we provided an open source code in order to further research and expand the model to fit to the desired needs.