\chapter{Improvements, applications and conclusion}

Finally after developing the model for a housing market with a supply shortage we have come across some observations, conclusions and possible improvements as future work. On this chapter we will deepen into these ideas. We will start by proposing a set of possible improvements and lastly present our conclusions over the project.

\section{Improvements}

While implementing and running simulations of the model we noticed that the time needed in order to implement the complete model exceeded the time planned for this paper and we left the Category 3 out of the implementation. For this reason we will start this section by exposing how the category 3 could be added to the model as well as expanding the plotting capabilities in order to plot over two variables in three dimensions and lastly implementing a utility function for the agents to help with the choice making.

Moreover at the end of the project we realised that in order for the simulation to run faster we would need to find a solution since each one took around  one and two hours to run. For this we will mention a proposal to increase the efficiency of the code as wee as improving the quality of the code by sharing methods across data classes.

\subsection{Category 3}

As mentioned in chapter 3 we implemented the model with C1 and C2 and left C3 aside. However the expansion to include it is straight forward. Firstly we would recommend to initialize the arrivals to the market in the same way as for C1 and C2. After initializing it we would then store them in a dictionary "landlords\_C3" the same as for C1 and c2. The change is that the matching procedure differs from the two other categories. 

The principle for the matching is that the agents is desperate to find accommodation or is about to perish and needs to diversify the way he is searching for accommodation. After a certain percentage of its time has been used he will then apply for accommodation in C3. Then he will be matched to options with the probability \textbf{pMC3} and either take the accommodation with probability \textbf{pTC3} or make use of an utility function to make a decision. The first option can be implemented analogously as in C2 but with lower probabilities. Fro the second option the utility function would need to be implemented, more on the in the next section.

As shown an expansion to include C3 is very much possible and the model could be expanded in a fast manner. However it would make sense to expand the model not only with a new category but also with the utility function in order to have a more dynamic apartment allocation. We will explain this concept in more detail next.

\subsection{Utility function}

When including C3 to the model the number of options an agent has or could have to choose accommodation from increases. In order to help the agent make the best decision for him we could develop a function that computes the utility an agent has on taking an apartment. The function we propose is a two tuple function. It computes the utility having as input the time the agent has waited (or still has before perishing) and the price (or integer value) of the apartment U = f(t, price). The more time passes the more willing is the agent to enter a higher category. Then with the used of the already known probabilities the agents gets matched as usual but when making a decision of which accommodation to take not only will the "taking probability" be consulted but also the utility \textbf{U} the agent gents from said apartment.

Moreover since the decision of taking the apartment or even applying to a new category can be now expanded to include the utility, the distribution of supply can be better tailored to a the desired housing market that will be implemented. In order to implement the utility function we would only need to add an attribute price/value to the landlord dictionary and the function f(t,price). The first one is very easy to implement as it only requires to add the extra attribute and we propose that this value can also be computed with the help of a exponential distribution for each category as we did with the maximum waiting times of the agents. This way we can modify the values of the apartments independently and are able to simulate them as similar or different as the scenario requires. This will make the decisions of the agents easier/tougher as we require them to be.

\subsection{3-D Plotting}

After having ran several simulations and observed the data we encountered cases where running over two variables would have been beneficial. This is the case since some variables are heavily linked together and this link could be better visualized by running them together over a 3-D plot. We implemented four modules capable of this. One to run over $\lambda$C2 and \textbf{pMC2}, the second one runs over $\lambda$C2 and $\lambda$C1 and the last one over $\lambda$C1 and \textbf{pTC1}.

The principal problem with this type of plotting is the time it takes for each simulation to finish. The estimated finish time to run a simulation over $\lambda$C1 from 0-10 and \textbf{pTC1} from 0-30 was 27 hours. Until the code isn't optimized, multi -threaded or ran in a data center, three dimensional plotting is not realistic under this model and hardware. The code is available and can be found in the source code.

\subsection{Efficiency}

As already mentioned, the time constraint was one of the biggest challenges for this paper. So an issue of great importance, was the time spent during the simulations. As an example a simulation with $\lambda$A = 10, $\lambda$C1 = 5 and $\lambda$C2 while computing the probability of \textbf{pTC1} from 0 to 30\% takes around two and a half hours. The hardware was a 1,4 GHz Quad-Core Intel Core i5 as CPU a Intel Iris Plus as GPU and 16GB of RAM. While tracking the usage of the hardware it was clear that it was being used very inefficiently. At all times only one core was being used and the GPU was barely used.

Our proposal to improve efficiency is to make use of the multi-threading libraries available in python. The approach here would be to divide the first main loop between threads. Taking the example of the used hardware and simulating over \textbf{pTC1} from 0 to 30. We have 4 cores that can individually run independent from each other as well 31 independent values from each other. We would then assign each thread a part of the variables to be simulated. Then the different cores would be able to run the simulation no precise order and at the end of the simulation order the results and pass them to the plotting methods. As an example lets assume that we divide the variables as follows with four threads: Thread\_1 computes from 0-8, Thread\_2 from 9-16, Thread\_3 from 17-23 and Thread\_4 24-31. This way we could theoretically reduce the computing time with the same hardware to under one hour.

Lastly we could use more of the GPU by delegating small arithmetic calculations to it and not to the CPU. An example of an easy arithmetic calculations that can be parallelized are the calculations of probabilities \textbf{pTC1} and \textbf{pMC2}. These probabilities are computed several times per iteration and would shorten the running time of the simulation if implemented to be computed in the GPU.

\subsection{Code quality}

For the purpose of quick prototyping we concentrated mainly in the functional side of the code and left the code quality for future work. There are code blocks that could be shared between modules.Each module computes a certain variable, lambda or waiting time. For example the module lambdaC1.py and lambdaC2.py computes over said lambdas in their respective categories. They use similar code blocks but it is not shared between them. The idea could is to centralize certain code blocks so they can be shared, thus reducing the code length, complexity and repetitive code adjustments.

\section{Applications}
The developed software could be expanded and improved as explained in the section above and thus could be used to simulate or approximate the market distribution of houses as well as the behaviour and choices of agents. The model can be filled with data from the housing and registry office and approximate a market as close as possible. There are several sectors that recollect this type of data with different purposes. The variables they would have to fill are the ones explained in the introduction of this paper. For example a government may use this type of model using their data to fill the variables of the simulation in order to observe the behaviour of their market and thus launch incentives to housing categories of the market in order to get to an equilibrium point and achieve the best possible growing and accommodation rate for the city. Moreover they can stop agents from going to other cities by accommodating them in the categories that the simulations recommend.

Furthermore this model offers a open source platform for further research and development and thus the model can be expanded, changed or adapted in order to use it for other similar markets. Some examples of this kind of markets behaviours are: job hunting (where there are many applicant but little jobs), organ transplantation (with  FCFS queue and sometimes lotteries and matching) as well as school allocation for agents (bests schools receive higher number of applications). This markets consists of a supply-demand chain where the demand exceeds the supply and agents apply for a chance of getting somethings out of the supply side. Nevertheless by expanding this model we can achieve a higher accuracy by simulating the housing market with existing data.

\section{Conclusion}

This paper aimed to research the housing market under a supply shortage by composing a model where agents and houses arrive to the market with a poison process and then code and compute several simulations of said model. Based on said simulations under the adoption of the model we developed, it can be concluded that agents will have strong inclination towards choosing housing in C1. This leads to C2 houses to have a higher average waiting time until they find an agent and will be left out of the market. The results also indicate that all markets under this model can be divided in three and all simulations passed through an equilibrium point where supply fulfilled all agents needs for accommodation. In the case of a market with a supply shortage we have shown that the market can be modeled and we can test what would happen if there were more apartments in C1 or C2 or with less agents in the market in order to meet the equilibrium point. This proved to be relevant information for government institutions that have to make incentive choices in order to stop people from leaving their territory because of lack of supply. This simulations bring a new overview and could potentially help to make more accurate decisions and incentives the places in supply that need help.


Lastly by providing an open source base code, model and documentation that can be easily expanded and molded to the user needs, we enabled further research in this sector. The code can be tailored to the needs of the user and could potentially even be molded to simulate other markets with this approach. With this and the achieved data result the paper achieved its purpose of making the housing market with supply shortage easier to visualize, understand and propose possible solutions for it.