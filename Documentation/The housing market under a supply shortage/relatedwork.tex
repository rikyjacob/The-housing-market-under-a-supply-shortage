\chapter{Related Work}

In the paper "The Public-Housing Allocation Problem" \cite{thakral2016public} from Harvard University they researched the housing distribution on agents in Pittsburgh. They approached the problem by simulating the housing market, by implementing a waiting queue combined with a stochastic model simulating the arrival and leaving of new agents. This model can be used to fit for markets with high volatility as the ones described earlier. In the case of implementing a utility function to make a choice between C2 and C3 we can rely on the paper "House Allocation with Existing Tenants" \cite{abdulkadirouglu1999house} from the University of Rochester where they introduced an utility function for the agents to simulate their willingness to received said accommodation.

The paper "Design of lotteries and wait-lists for affordable housing allocation" \cite{arnosti2020design} from Management Science defines the use and implementation of waiting queues in a simulation context. This paper will be used as baseline for coding the matching algorithms for the platforms. The paper "Dynamic assignment of objects to queuing agents" \cite{bloch2017dynamic} from the American Economic Association implements a FCFS queue to allocate resources to agents. This can be easily transformed and used in our model. Lastly the paper "Dynamic matching in overloaded waiting lists" \cite{leshno2019dynamic} from Columbia Business School researches queues in matching objects to agents in markets with overload. We can implement this for our market with high demand. It uses stochastic processes to model the arrival of new agents.

Finally the paper "Matching with stochastic arrival" \cite{thakral2019matching} from the American Economic Association analyses the usage and implementation of a stochastic model to simulate the agents arrival to the market.